%%%%%%%%%%%%%%%%%%%%%%%%%%%%%%%%%%%%%%%%%
% Structured General Purpose Assignment
% LaTeX Template
%
% This template has been downloaded from:
% http://www.latextemplates.com
%
% Original author:
%  Ted Pavlic (http://www.tedpavlic.com)
% Modified by:
%  Joe Del Rocco (https://joe.delrocco.org)
%%%%%%%%%%%%%%%%%%%%%%%%%%%%%%%%%%%%%%%%%

%----------------------------------------------------------------------------------------
%  PACKAGES AND CONFIGURATION
%----------------------------------------------------------------------------------------

\documentclass{article}
\usepackage{geometry}
\usepackage{fancyhdr} % For custom headers
\usepackage{lastpage} % To determine the last page for the footer
\usepackage{extramarks} % For headers and footers
\usepackage[most]{tcolorbox} % For problem answer sections
\usepackage{graphicx} % For inserting images
\usepackage{xcolor} % For link coloring
\usepackage[hidelinks]{hyperref} % For URL links (no box or color name)
\usepackage{bm}
\usepackage{amsmath, amssymb, amscd, amsthm, amsfonts}
\usepackage[ansinew]{inputenc} 
\usepackage{xcolor}
\usepackage{mathtools}
\usepackage{wrapfig}
\usepackage{tabularx}
\usepackage{multirow}
\usepackage{booktabs}
\usepackage[labelfont=bf]{caption}
\usepackage{subcaption}
\usepackage{enumerate}
\usepackage[ruled,vlined]{algorithm2e}
\usepackage{bigints}
\usepackage[edges]{forest}
\usepackage{wrapfig}
\usepackage[toc,page]{appendix}

\newtheorem{theorem}{Theorem}
%\theoremstyle{definition}
\newtheorem{definition}{Definition}[section]

\newtheorem{lemma}[theorem]{Lemma}

% Margins
\geometry{
a4paper,
tmargin=1in,
bmargin=1in,
lmargin=1in,
rmargin=1in,
textwidth=6.5in,
textheight=9.0in,
headsep=0.25in
}

% Header and footer
\pagestyle{fancy}
\lhead{\myName} % Top left header
\chead{\myAssignement} % Top center header
\rhead{\HomeworkNumber} % Top right header
\lfoot{\lastxmark} % Bottom left footer
\cfoot{\thepage} % Bottom center footer
%&\rfoot{Page\ \thepage\ of\ \pageref{LastPage}} % Bottom right footer
\renewcommand\headrulewidth{0.4pt} % Size of the header rule
%\renewcommand\footrulewidth{0.4pt} % Size of the footer rule

%% Other configurations
\setlength\parindent{0pt} % Removes all indentation from paragraphs
\setlength\parskip{1pt} % Ensures paragraphs are still recognizable as such
\setcounter{secnumdepth}{0} % Removes default section numbers
%\setcounter{tocdepth}{3} % Sets depth of table of contents
\linespread{1.1}

% Template values
\newcommand{\myLogo}{figures/logo1.pdf}
\newcommand{\myName}{Marcello Bullo}
\newcommand{\myJobTitle}{MSc. students}
\newcommand{\myStudentID}{1204533}
\newcommand{\myCompany}{University of Padova}
\newcommand{\myLocation}{1701 Lincoln Blvd, San Francisco, CA}
\newcommand{\myURL}{www.starfleet.edu}
\newcommand{\myEmail}{marcello.bullo@studenti.unipd.it}
\newcommand{\myCourse}{Optimization}
\newcommand{\mySection}{Spring 2325}
\newcommand{\myTeacher}{Francesco Rinaldi}
\newcommand{\HomeworkNumber}{Final Project}
\newcommand{\myAssignement}{Accelegrad and UnixGrad: a Comparison Analysis}
\newcommand{\myDueDate}{19th February\ 2021}

%----------------------------------------------------------------------------------------
%  DOCUMENT STRUCTURE (MACROS & ENVIRONMENTS)
%----------------------------------------------------------------------------------------

% Colored links macro
\newcommand{\hrefcol}[3] {\href{#1}{\textcolor{#3}{#2}}}

% Creates a counter to keep track of the number of problems
\newcounter{homeworkProblemCounter}

% Macro for custom title page signature header
\newsavebox{\myTitleSignature}
\sbox{\myTitleSignature}{%
%\begin{tabular*}{\textwidth}{@{}lr@{\extracolsep{0.125in}}l@{}}%
\begin{tabular*}{\textwidth}{@{}l|@{\extracolsep{0.125in}}l@{}}%
\parbox{4.25in}{\raggedright{\includegraphics[width=0.45\linewidth]{\myLogo}}} &%
%\parbox{1.8in}{\raggedleft{\includegraphics[width=0.16\linewidth]{Figures/dei_logo.pdf}}} &%
\parbox[c][]{2.2in}{{\textbf{\myName} \par}
					{\textbf{Francesco Mandruzzato} \par}
                    {\small \myJobTitle \par}
                    {\small \myCompany \par}
                    {\small ID: \myStudentID - 1204532 \par}
                    {\small \hrefcol{mailto:\myEmail}{\myEmail}{black} \par}
                    {\small \hrefcol{mailto:francesco.mandruzzato.1@studenti.unipd.it}{francesco.mandruzzato.1@studenti.unipd.it}{black}} \par}
\end{tabular*}}

% Folder tree style
%\definecolor{folderbg}{RGB}{124,166,198}
%\definecolor{folderborder}{RGB}{110,144,169}
%
%\def\Size{4pt}
%\tikzset{
%  folder/.pic={
%    \filldraw[draw=folderborder,top color=folderbg!50,bottom color=folderbg]
%      (-1.05*\Size,0.2\Size+5pt) rectangle ++(.75*\Size,-0.2\Size-5pt);  
%    \filldraw[draw=folderborder,top color=folderbg!50,bottom color=folderbg]
%      (-1.15*\Size,-\Size) rectangle (1.15*\Size,\Size);
%  }
%}


 \definecolor{folderbg}{RGB}{124,166,198}
\definecolor{folderborder}{RGB}{110,144,169}
\newlength\Size
\setlength\Size{4pt}
\tikzset{%
  folder/.pic={%
    \filldraw [draw=folderborder, top color=folderbg!50, bottom color=folderbg] (-1.05*\Size,0.2\Size+5pt) rectangle ++(.75*\Size,-0.2\Size-5pt);
    \filldraw [draw=folderborder, top color=folderbg!50, bottom color=folderbg] (-1.15*\Size,-\Size) rectangle (1.15*\Size,\Size);
  },
  file/.pic={%
    \filldraw [draw=folderborder, top color=folderbg!5, bottom color=folderbg!10] (-\Size,.4*\Size+5pt) coordinate (a) |- (\Size,-1.2*\Size) coordinate (b) -- ++(0,1.6*\Size) coordinate (c) -- ++(-5pt,5pt) coordinate (d) -- cycle (d) |- (c) ;
  },
}
\forestset{%
  declare autowrapped toks={pic me}{},
  pic dir tree/.style={%
    for tree={%
      folder,
      font=\ttfamily,
      grow'=0,
    },
    before typesetting nodes={%
      for tree={%
        edge label+/.option={pic me},
      },
    },
  },
  pic me set/.code n args=2{%
    \forestset{%
      #1/.style={%
        inner xsep=2\Size,
        pic me={pic {#2}},
      }
    }
  },
  pic me set={directory}{folder},
  pic me set={file}{file},
}


%----------------------------------------------------------------------------------------
%   TITLE PAGE
%----------------------------------------------------------------------------------------
\begin{document}

% Blank out the traditional title page
\title{\vspace{-1.2in}} % no title name
\author{} % no author name
\date{} % no date listed
\maketitle % makes this a title page

% Use custom title macro instead
\usebox{\myTitleSignature}
\vspace{0.3in} % spacing below title header

% Assignment title
{\centering \huge \textsc{\HomeworkNumber}:\\ \myAssignement \par}
{\centering \noindent\rule{4.25in}{0.1pt} \par}
\vspace{0.in}
{\centering \myCourse, \myTeacher \par}
{\centering \myDueDate \par}
%{\centering Prepared w/ \LaTeX \par}
\vspace{0.3in}
%{\centering \noindent\rule{\textwidth}{0.6pt} \par}
%\vspace{-0.17in}
%{\centering \noindent\rule{\textwidth}{0.5pt} \par}

% Table of Contents
%\tableofcontents
%\newpage


\section{Introduction}

\section{Accelegrad}

\subsection{Previous related work}

The Accelerated Gradient Descent method devised by Nesterov is a popular optimization technique which provides faster convergence rate $\mathcal{O}(1/T^2)$ in the smooth setting, guaranteeing standard rate $\mathcal{O}(1/\sqrt T)$ for a general convex objective function $f$. The structure of the algorithm is divided in two different steps:

\begin{enumerate}
  \item \textbf{Extrapolation step}: We move along the direction of the difference between the last two iterates:
  	  \begin{equation}
  			y_k = x_k + \beta_k(x_k - x_{k-1}),
	  \end{equation}
	  were $\beta_k$ is chosen depending on the properties of $f$. Ideally this parameter regulates the amount of information that we want to keep regarding past iterates.
  \item \textbf{Gradient step} We performe a step likewise the GD algorithm in $y_k$ to get the next point $x_{k+1}$:
  	  \begin{equation}
  			x_{k+1} = y_k - \alpha_k \nabla f(y_k),
	  \end{equation} 
	  where $\alpha_k = 1/L$ 
\end{enumerate}

Despite its efficiency, Accelerated Gradient Descent is inappropriate for handling noisy feedback because of the choice of the fixed step size $\alpha_k$, moreover acceleration requires the knowledge of the objective's smoothness.

One of the main popular adaptive first order methods comes with AdaGrad \cite{adagrad}. Informally, AdaGrad exploits the geometry of the dataset in the sense that the adaptation of the step size $\alpha_k$ vary with the dimension considered, in other words, it associates a small step size to frequently occurring features and a large step size to infrequent features. This adaptation is encoded in its update rule:
\begin{equation}
  x_{t+1} = \Pi_K^{G_t^{1/2}} (x_t - \eta G_t^{-1/2} g_t ),
\end{equation}

where $G_t = \sum_{\tau=1}^t g_t g_t^T $ is the matrix of the previous subgradients, and $\Pi_K$ is the projection of the gradient into the set $K$. 
With this adaption AdaGrad may handle noisy feedback, however there is no guarantee that AdaGrad can ensure acceleration, moreover it was unknown whether AdaGrad is able to exploit the smoothness in order to converge faster. Considering the advantage of handling noisy feedback to ensure a solid acceleration, Accelegrad takes inspiration from the update rule of AdaGrad. 

Regarding acceleration, an elegant way of interpreting it comes from \cite{linearcup} and we provide the high level concept which is useful to understand the algorithm steps. 

In \emph{Allen-Zhu} and \emph{Orecchia}, they combine standard Gradient Descent and Mirror Descent to yield a new and simple accelerated gradient method for smooth convex optimization problems.
Given an $L$-smooth function $f$ we are able to find a quadratic upper bound of the form:
$$
f(y) \leq f(x) + \langle \nabla f(x),y-x \rangle + \frac{L}{2} \Vert y - x \Vert^2. \ \ \ \ \forall y
$$
This is particularly useful to find an upper bound to the maximal objective decrease given as:
$$
f(x_{k+1})-f(x_k) \leq - \frac{1}{2L}\Vert \nabla f(x_k) \Vert^2.
$$
From this point it should be clear that the larger the magnitude of the gradient the better may be the convergence speed of Gradient Descent. Mirror Descent instead peeks a uniformly distributed sequences of points $(x_1,\dots,x_n)$ and it combines them to construct a stronger lower bound to the function $f$ which is not provided by the standard Gradient Descent, in particular this lower bound is obtained as: 
$$
f(y) \geqslant \frac{1}{n} \sum_{t=1}^{n} f(x_t) + \frac{1}{n} \sum_{t=1}^{n} \langle \nabla f(x_t), y-x_t \rangle \ \ \ \forall y.
$$
To construct a good upper bound instead we consider the mean of the queried points and by the convexity property we get $f(\hat x) \leq \frac{1}{n} \sum_{t=1}^n f(t_i)$. Combining this with the previous result we get:
$$
f(\hat x) - f(y) \leq \frac{1}{n} \sum_{t=1}^{n} \langle \nabla f(x_t), y-x_t \rangle, 
$$ 
which is known as the regret bound $R_n(y)$. Intuitively since we want to keep as small as possible the regret of an algorithm, contrary to gradient descent we need to receive small gradients, in fact it is possible to prove that Mirror Descent converge faster with small gradient magnitudes as it's converge rate is $\mathcal{O}(\rho^2/\epsilon^2)$ where $\rho^2$  is equal to the average magnitude of the queried gradients.

In light of this consideration, they propose a linear combination of the two algorithms at each step, in particular following Nesterov, given the gradient step $y_k$ and the mirror step $z_k$ the next step will be given by:
\begin{equation}
  x_{k+1} = \alpha z_k + (1-\alpha) y_k,
\end{equation}

where $\alpha_k$ is a tuneable parameter.
It is possible to prove that this technique provides the same convergence guarantees of the accelerated gradient method of Nesterov in a nicer and simpler way, for this reason Accelegrad incorporates the notion of acceleration taking inspiration from this linear coupling framework.
  
\subsection{Accelegrad algorithm setting}

In the Accelegrad paper \cite{accelegrad}, there is no assumption about the smoothness parameter $\beta$ (See Def.\ref{b-smooth} of smoothness), but they assumed to be given a bound between some initial point $x_0$ and a global minimum of $f$ for a given compact convex set $K$ which contains this global minimizer. This distance is bounded by the diameter of $K$ i.e. $D = \max_{x,y \in K} \Vert x-y \Vert $. Finally they assume that the function $f$ is G-Lipschitz for which we provide the definition in (Def.\ref{g-lipschitz}).

First, we discuss the offline setting where we have always access to the exact gradients of $f$, the algorithm is presented in Algo.\ref{alg:accelegrad-algo}. As we previously said, Accelegrad takes inspiration from \cite{linearcup} and linearly couples between two sequences $\{y_t\}_t$ and $\{z_t\}_t$ into $\{x_{t+1}\}_t$. These two sequences are updated in two different manners, the former ($y_{t+1}$), takes a step starting from $x_{t+1}$, the latter ($z_{t+1}$), takes a step starting from $z_t$. Both sequences are updated with the same step size but for $z_{t+1}$ the gradient is scaled by a factor of $\alpha_t$ and the final result is projected into $K$. The step size is defined taking inspiration from AdaGrad:
\begin{equation}
  \eta_t = \frac{2D}{(G^2 + \sum_{\tau = 0}^t \alpha_{\tau}^2 \Vert g_t \Vert^2)^2},
\label{learning-1}
\end{equation}
where the only difference are the importances weights $\alpha_t$ defined as:
\begin{equation}
  \alpha_t = \begin{cases}
				1 \ \ \ \ \ \ 0 \leq t \leq 2\\
				\frac{1}{4}(t+1) \ \ \ \ \ \ \ t \geqslant 3
			 \end{cases},
\label{alpha}
\end{equation}
which increase with $t$, putting emphasis on recent queries of the gradient. As we are considering the smooth setting it is not required the function to be $G$-Lipschitz which usually becomes important in the non smooth setting.



\begin{algorithm}
    \SetKwInOut{Input}{Input}
    \SetKwInOut{Output}{Output}

    \Input{\#Iterations T, $x_0 \in K$, diameter $D$, weights $\alpha_{t \in T}$, learning rate $\eta_t$} 
    Set $y_0 = x_0 = z_0$ \\
    \For{$t=1\dots T$}
      {
      	Set $\tau_t = 1/\alpha_t$ \\
      	Update:
			$$ x_{t+1} = \tau_t z_t + (1-\tau_t) y_t, \ \ g_t = \nabla f(x_{t+1}) $$
			$$ z_{t+1} = \Pi_K (z_t - \alpha_t \eta_t g_t) $$
			$$ y_{t+1} = x_{t+1} - \eta_t g_t $$
      }
    \Output{$ \hat y_T \propto \sum_{t=0}^{T-1} \alpha_t y_{t+1} $}
    \caption{Accelerated Adaptive Gradient Method (AcceleGrad)}
    \label{alg:accelegrad-algo}
\end{algorithm}

\newpage

For the smooth setting Accelegrad requires a number of iterations to get a solution within $\epsilon$ of $\mathcal{O}(\frac{c}{\epsilon})^{\frac{1}{2}}$ with $c = DG + \beta D^2log(\beta D/G)$, which means that the algorithm is able to converge as fast as the Accelerated gradient method asymptotically. Actually, since we are dealing with smooth functions we don't need to consider $G$, which means that the constant $c$ is actually equal to $c = \beta D  log(\beta D /\Vert g_0 \Vert)$.

Considering the non smooth setting the previous algorithm is able to ensure standard sublinear convergence rate of $\mathcal{O}(1/\sqrt T)$. With this rate of convergence Accelegrad requires a number of iterations to get a solution within $\epsilon$ of..., this result is also true for the stochastic optimization setting. In the Appendix we provide a short version of the proofs regarding the convergence rate for these three cases. 




%Typically GD is analysed w.r.t a convex L-smooth function $f$, in this setup we get a quadratic upper bound of the form: 
%$$
%f(y) \leq f(x) + \langle \nabla f(x), y-x \rangle + \frac{L}{2} \Vert y-x \Vert^2. \forall y
%$$
%Under this assumption, using the GD update rule and a straightforward property of the convexity we get:
%
%$$
%x_{k+1} = x_k - \alpha \nabla f(x_k) \ \ \ (GD \ update \ rule)
%$$
%$$
%f(x) - f(y) - \nabla f(y)^T(x-y) \leq  \frac{L}{2} \Vert y-x \Vert^2, \forall x,y \in \mathbb{R}^n \ \ \ (Convexity \ property)
%$$
%$$
%f(x_{k+1}) - f(x_k) \leq \nabla f(x_k)^T(x_k - \alpha \nabla f(x_k) - x_k ) + \Vert x_k - \alpha \nabla f(x_k) - x_k \Vert^2 = -(1-\frac{L\alpha}{2})\alpha \Vert \nabla f(x_k) \Vert^2,
%$$
%
%which is maximised for $\alpha = 1/L$. The final term provides an upper bound on the gradient 

\section{UnixGrad}
\subsection{Previous related works}
\subsection{UnixGrad algorithm setting}














\section{Comparison Results}

In this section we provide a comparison between AcceleGrad and UnixGrad. We test the two algorithm with some SVM problems taken from LIBSVM. In particular we worked with \emph{breast-cancer} and \emph{adult} datasets. We try to minimize the L2-regularized squared hinge loss using random mini-batches of size 5, with random weights initialization. For both UnixGrad and Accelegrad we measure the performance with respect to the average iterate which results in a more stable behaviour. 


\section{Online to Batch conversion}

Online learning is the process of answering a sequence of questions given knowledge of the correct answers to the previous ones. Given a sequence of questions $S=(x_1, \dots, x_t)$, for each one of them sequentially our algorithm makes a prediction and then after the prediction the correct answer is revealed incurring in a loss. The goal of the learner is to minimize the loss also exploiting past information. If there is no correlation between samples, learning is hopeless.

Intuitively, for each sample passed to the learner at stage $t$, the learner chose an hypothesis $h$ from the hypothesis class $H$ and predicts the label associated with the sample. For this reason we can define a regret function which measure "how sorry" the learner is not to have followed the best predictor $h^*$. Formally when running on a sequence of $T$ examples:
$$
	R_T(h^*) = \sum_{t=1}^{T} l(p_t,y_t) - \sum_{t=1}^{T} l(h^*(x_t),y_t),
$$
where $p_t$ is the prediction obtained by $h$ at stage $t$, and the regret w.r.t the hypothesis class $H$ is defined as:
$$
	R_T(H) := \max_{h^* \in H} R_T(h^*).
$$
Usually we look for an algorithm which converge sub-linearly to the lowest possible regret.

What we described in short say that an online learning algorithm wants performance close to the single best hypothesis chosen in hindsight given all the data.

Different from this practice, an online to batch converter needs to construct a model based on the set of modules generated by the online learner, this can be done in different ways but one of the most classical way is to take the average of these models. This conversion is needed because the individual iterates of an online algorithm do not come with individual guarantees. The standard online to batch conversion algorithm format is shown in Algo.\ref{alg:onlinetobatch}. In particular since the algorithm outputs an average of the iterates, we can apply Jensen's inequality to show the following:
$$
\mathbb{E}[L(\hat w) - L(x^*)] \leq \frac{\mathbb{E}[R_T(x^*)]}{T}, 
$$
and as long as the algorithm obtains sublinear regret the left side part of the inequality will approach zero an average.

\begin{algorithm}
    \SetKwInOut{Input}{Input}
    \SetKwInOut{Output}{Output}

    \Input{\#Iterations $T$, Params $S$, Cost function $l$, Algorithm $A$}
    \For{$t=1\dots T$}
      {
		let $w_t$ be the prediction of $A$, provide the cost function $l(w_t,g_t)$ to $A$
		where $\mathbb{E}[g_t] = \nabla F(w_t)$
      }
    \Output{$\hat w = \frac{1}{T} \sum_{t=1}^T w_t$}
    \caption{Online to batch conversion }
    \label{alg:onlinetobatch}
\end{algorithm}

In \cite{cutkosky} to avoid the guarantees problems of single iterates it is provided a black box "Anytime" Online to Batch conversion algorithm. The "Anytime property" comes from the fact that the last iterate is always a good estimate of $x^*$. The algorithm is very similar to the classical online to batch previously shown. The key difference is that they evaluate the stochastic gradient oracle at $x_t$ rather than the iterates provided by the algorithm $A$, furthermore they incorporate the importance weights $\alpha_t$ which are useful to achieve faster convergence rates on smooth losses. 

\begin{algorithm}
    \SetKwInOut{Input}{Input}
    \SetKwInOut{Output}{Output}

    \Input{\#Iterations $T$, Online Algorithm $A$ with convex domain $D$, Non-negative weights $\alpha_t$ with $\alpha_t > 0$}
    \For{$t=1\dots T$}
      {
		Get $x_t = \frac{\sum_{i=1}^t \alpha_i w_i}{\sum_{i=1}^t \alpha_i} $ \\
		Play $x_t$ and receive subgradient $g_t$ \\
		Send $l_t(x) = \langle \alpha_t g_t,x\rangle$ to $a$ as the $t$th loss \\
		Get $w_{t+1}$ from $A$
      }
    \Output{$x_T$}
    \caption{Anytime Online to batch conversion }
    \label{alg:anytime}
\end{algorithm}

Different from the standard online to batch conversion, this algorithm is able to achieve the following guarantees for every $x^* \in D$:
$$
\mathbb{E}[L(x_T) - L(x^*)] \leq \frac{\mathbb{E}[R_T(x^*)]}{\sum_{t=1}^T \alpha_t}, 
$$ 

in particular this bound is valid also for loss functions with some known non-linearity. In the Appendix we provide a proof sketch of this bound.



\section{General definitions}


\begin{definition}
	A function $f: \mathbb{R}^n \rightarrow \mathbb{R}$ is $\beta$-smooth if:
	$$
		f(y) \leq f(x) + \nabla f(x)^T(y-x) + \frac{\beta}{2} \Vert x-y \Vert^2, \forall x,y \in \mathbb{R}^n
	$$
\label{b-smooth}
\end{definition}

\begin{definition}
	$f$ is G-Lipschitz if $\forall w \in \Omega$ we have $\Vert \nabla f_i(w) \Vert \leq G$
\label{g-lipschitz}
\end{definition}

\begin{definition}
	Given a real-valued convex function $f$ with elements $x_1, \dots x_n$ in its domain and positive weights $a_i$, Jensen's inequality is stated as:
	$$
		f\bigg(\frac{\sum a_i x_i }{\sum a_i}\bigg) \leq \frac{\sum a_i f(x_i)}{\sum a_i}
	$$
\label{jensen}
\end{definition}

\bibliography{biblio}
\bibliographystyle{ieeetr}


\newpage
\begin{appendices}
\section{Accelegrad}

\begin{theorem}
Assume that $f$ is convex and $\beta$-smooth. Let $K$ be a convex set with bounded diameter D, and assume there exists a global minimizer for $f$ in $K$. Then Algorithm \ref{alg:accelegrad-algo} with weights equal to Equation \ref{alpha} and learning rate as in Equation \ref{learning-1} ensures:
$$
f(\hat y_t) - f(x^*) \leq \mathcal{O} \bigg (\frac{DG + \beta D^2 log(\beta D / G)}{T^2}\bigg)
$$
\label{Theorem-1-acc}
\end{theorem}

Since the proof of this Theorem is very long, we provide the most meaningful parts which guides us into deriving the upper bound on the error. 

\begin{proof}
Our objective is to find a bound of the left-side in order to precisely state the convergence rate of Accelegrad in the smooth case. Given the fact that the algorithm outputs a weighted average of gradients we may apply the Jensen Inequality defined in \ref{jensen} as:
$$
	f(\hat y_t) - f(x^*)	 \leq \frac{1}{\sum_{t=0}^{T-1} a_i} \sum_{t=0}^{T-1} a_i ( f(y_{t+1}) - f(x^*) )
$$

Given the fact that $\sum_{t=0}^{T-1} a_i \geqslant \Omega(T^2) $ it is sufficient to bound the numerator by a constant in order to get the sublinear converge rate $\mathcal{O}(C/T^2)$ and the proof is concentrated precisely at this point. 

Without re-writing the proof of Lemma described in the paper, we can bound our term of interest by:

$$
	\sum_{t=0}^{T-1} a_i ( f(y_{t+1}) - f(x^*) )	 \leq (\alpha_t^2 - \alpha_t)(f(y_t)-f(y_{t+1})
$$
$$
+ \frac{\alpha_t^2}{2}\bigg(\beta_t - \frac{1}{\eta_t}\bigg) \Vert y_{t+1} - x_{t+1} \Vert^2 
$$
$$ 
+ \frac{1}{2\eta_t}(\Vert z_t - z \Vert^2 - \Vert z_{t+1} - z \Vert^2).
$$

Each term of the sum is in turn bounded, providing a more elegant overall bound:
$$
\frac{1}{2} \sum_{t=0}^{T-1} a_i ( f(y_{t+1}) - f(x^*) )	 \leq \underbrace{\frac{D^2}{2 \eta_{T-1}} - \frac{1}{4} \sum_{t=\tau_*+1}^{T-1} \eta_t \alpha_t^2 \Vert g_t \Vert^2}_\text{(*)} + \underbrace{\frac{\beta}{2} \sum_{t=0}^{\tau_*} \eta_t^2 \alpha_t^2 \Vert g_t \Vert^2}_\text{(**)}.
$$
Now by fixing the learning rate equal to the value in Equation.\ref{learning-1}, we provide a final bound which is constant and we conclude the proof. These steps requires many computations and Lemmas that we skip in order to provide the higher point of view. 

The final result is of the form:
$$
\frac{1}{2} \sum_{t=0}^{T-1} a_i ( f(y_{t+1}) - f(x^*) )	 \leq DG/4 + 2\beta D^2 + 2\beta D^2 (1 + 2log(4\beta D / D)
$$
Combining this result with our initial inequality we get:
$$
f(\hat y_t) - f(x^*) \leq \frac{1}{\sum_{t=0}^{T-1} a_i} \sum_{t=0}^{T-1} a_i ( f(y_{t+1}) - f(x^*) ) 
$$
$$
\leq \frac{DG/2 + 8\beta D^2(1 + log(4\beta D / D)}{T^2 / 32}
$$
$$
\approx \mathcal{O} \bigg( \frac{DG + \beta D^2 log(4\beta D / D)}{T^2} \bigg),
$$
where we used the fact that $\sum_{t=0}^{T-1} a_i \geqslant T^2 / 32 $, which conclude the proof for the smooth case.
\end{proof}

\begin{theorem}
Assume that $f$ is convex and $G$-Lipschitz. Let $K$ be a convex set with bounded diameter D, and assume there exists a global minimizer for $f$ in $K$. Then Algorithm \ref{alg:accelegrad-algo} with weights equal to Equation \ref{alpha} and learning rate as in Equation \ref{learning-1} ensures:
$$
f(\hat y_t) - f(x^*) \leq \mathcal{O} \bigg ( GD \sqrt{log T} / \sqrt{T} \bigg)
$$
\label{Theorem-2-acc}
\end{theorem}

\begin{proof}
The proof follows the same considerations of the previous one, this time we need to prove that $\sum_{t=0}^{T-1} a_i ( f(y_{t+1}) - f(x^*) )$ is bounded by $\mathcal{O}(T^{3/2})$. As in the previous case it is possible to prove that:
$$
\alpha_t (f(y_{t+1}) - f(x^*)) \leq \eta_t \alpha_t^2 \Vert g_t \Vert^2 + \eta_t \alpha_t^2 \Vert g_t \Vert + \frac{1}{2 \eta_t} ( \Vert z_t^2 - z \Vert - \Vert z_{t+1}^2 - z \Vert) 
$$
$$
+ (\alpha_t^2 -\alpha_t)(f(y_t)-f(y_{t+1})).
$$
As before, we can bound each term of this sum and the combine the results to conclude our proof.
\end{proof}

\begin{theorem}
Assume that $f$ is convex and $G$-Lipschitz. Let $K$ be a convex set with bounded diameter D, and assume there exists a global minimizer for $f$ in $K$. Assume that we invole Algorithm \ref{alg:accelegrad-algo} with weights equal to Equation \ref{alpha} and learning rate as in Equation \ref{learning-1} ensures, but this time we provide it with noisy gradient estimates, then our algorithm ensures:
$$
\mathbb{E}[f(\hat y_t)] - f(x^*) \leq \mathcal{O} \bigg ( GD \sqrt{log T} / \sqrt{T} \bigg)
$$
\label{Theorem-2-acc}
\end{theorem}

\begin{proof}
In order to provide the proof of this theorem we assume that upon querying a first order oracle with a point $x$, we receive a bounded and unbiased gradient estimate $\tilde g$ such that $\mathbb{E}[\tilde g | x] = \nabla f(x)$ with $\Vert \tilde g \Vert \leq G$	. As in the previous theorems we need to find an upper bound for $\mathbb{E}[f(\hat y_t)] - f(x^*)$ to obtain that convergence rate. Obviously this bound is a bit different from the previous ones but the concept is the same.  

\end{proof}

\begin{theorem}
Suppose $g_1, \dots, g_t$ satisfy $\mathbb{E}[g_t|x_t] \in \delta L(x_t)$ for some objective function $L$ and $g_t$ is independent of all other quantities given $x_t$. Let $R_T(x^*)$ be a bound on the linearized regret of the Anytime Online to Batch algorithm. 
$$
R_T(x^*) \geqslant \sum_{t=1}^T \langle \alpha_t g_t, w_t - x^* \rangle 
$$
Then for all $x^* \in D$, the algorithm guarantees:
$$
\mathbb{E}[L(x_t)-L(x^*)] \leq \mathbb{E}\bigg[\frac{R_T(x^*)}{\sum_{t=1}^T \alpha_t}\bigg]
$$
Furthermore, suppose that $D$ has a diameter $B$ and $\Vert g_t \Vert_{*} \leq G$ with probability 1 for some $G$. Then with probability at least $1-\delta$:
$$
L(x_T)-L(x^*) \leq \frac{R_T(x^*)+2BG \sqrt{\sum_{t=1}^T \alpha_t^2 log(2/\delta)}}{\alpha_{1:T}}
$$
\label{Ashokproof-1}
\end{theorem}

\begin{proof}
In order to prove this theorem we observe that $\alpha_t(x_t-w_t) = \alpha_{1:t-1}(x_{t-1}-x_t)$, where $\alpha_{1:t-1} = \sum_{i=1}^{t-1} \alpha_i$. First we use the property of convexity to bound $\mathbb{E}[\alpha_t (L(x_t)-L(x^*))]$with $\mathbb{E}[\sum_{t=1}^T \alpha_t \langle g_t,x_t - x^* \rangle ]$, then using the previous observation in conjunction with some simple math adjustments we obtain:
$$
\mathbb{E}[\alpha_t(L(x_t)-L(x^*))] \leq \mathbb{E}\bigg[\sum_{t=1}^T \alpha_{1:t-1} (L(x_{t-1})-L(x_t)) \bigg],
$$
To conclude the proof of this theorem is sufficient to subtract $\mathbb{E}[\sum_{t=1}^T \alpha_t L(x_t)]$ from both sides and then apply the telescope series to obtain:
$$
\mathbb{E}[\alpha_{1:T} L(x_T)-\alpha_{1:T} L(x^*)] \leq \mathbb{E}[R_T(x^*)]
$$
$$
\mathbb{E}[L(x_T)-L(x^*)] \leq \mathbb{E}\bigg[\frac{R_T(x^*)}{\alpha_{1:T}}\bigg].
$$
Furthermore, let's define $H_{t-1}$ as the history $g_{t-1},x_{t-1},\dots,g_1,x_1$ and define $G_t = \mathbb{E}[g_t | H_{t-1},x_t,w_t]$ which is the expected value of the subgradient in relation to the past history. Let's define $\epsilon_t = \alpha_t \langle G_t, w_t - x^* \rangle - \langle g_t, w_t - x^* \rangle$ since $g_t$ is independent of all other quantities given $x_t$, we have that $\mathbb{E}[\epsilon_t | H_{t-1},x_t,w_t]=0$. If we sum all over the t's we obtain:
$$
\sum_{t=1}^T \epsilon_t = \sum_{t=1}^T  \alpha_t \langle \underbrace{G_t}_\text{($\leq G$)},\underbrace{w_t-x^*}_\text{($\leq B$)} \rangle - \sum_{t=1}^T  \alpha_t \langle \underbrace{g_t}_\text{($\leq G$)},\underbrace{w_t-x^*}_\text{($\leq B$)} \rangle,
$$
where $ |\epsilon_t| \leq 2\alpha_t BG$ with probability 1, and using the Azuma-Hoeffding bound with probability at least $1-\delta$ we get $\sum_{t=1}^T \epsilon_t \leq 2BG \sqrt{\sum_{t=1}^T \alpha_t^2 log(2/\delta)} $. This bound is useful to conclude our proof, the steps are similar to the previous part of the proof and with some simple math adjustments we obtain the final result of the theorem.  
\end{proof}




\section{Unixgrad}
\subsection{Regret-to-rate conversion}
In order to provide the mathematical analysis of the convergence rate of UnixGrad in the different optimization scenarios, classical tools in the online learning literature are taken into account an the convergence analysis is performed in the sense of bounding "weighted regret". Hence, a conversion strategy is adopted to directly translate the weighted regret to convergence rate.

\begin{lemma}[Regret-to-rate conversion]
\label{regret}
Consider weighted average 
$$\bar{x}_{t} =\frac{\alpha_{t} x_{t}+\sum_{i=1}^{t-1} \alpha_{i} x_{i}}{\sum_{i=1}^{t} \alpha_{i}}.$$
Let $R_{T}\left(x_{*}\right)=\sum_{t=1}^{T} \alpha_{t}\left\langle x_{t}-x_{*}, g_{t}\right\rangle$ denote the weighted regret after T iterations, $\alpha_{t}=t$ and $g_{t}=\nabla f\left(\bar{x}_{t}\right)$. Then,
$$
f\left(\bar{x}_{T}\right)-f\left(x_{*}\right) \leq \frac{2 R_{T}\left(x_{*}\right)}{T^{2}}
$$
\begin{proof}
	
	Proof. Let's define $A_{t}=\sum_{i=1}^{t} \alpha_{i} .$ Then, by definition, we could express $x_{t}$ as

\begin{equation}
\label{xt}
x_{t}=\frac{A_{t}}{\alpha_{t}} \bar{x}_{t}-\frac{A_{t-1}}{\alpha_{t}} \bar{x}_{t-1}
\end{equation}

Then, use Eq.~(\ref{xt}) and replace $g_{t}$ by $\nabla f\left(\bar{x}_{t}\right)$ in the weighted regret expression, i.e.
\begin{equation*}
\begin{aligned}
\sum_{t=1}^{T} \alpha_{t}\left\langle x_{t}-x_{*}, \nabla f\left(\bar{x}_{t}\right)\right\rangle &=\sum_{t=1}^{T} \alpha_{t}\left\langle\frac{A_{t}}{\alpha_{t}} \bar{x}_{t}-\frac{A_{t-1}}{\alpha_{t}} \bar{x}_{t-1}-x_{*}, \nabla f\left(\bar{x}_{t}\right)\right\rangle \\
&=\sum_{t=1}^{T} \alpha_{t}\left\langle\frac{A_{t}}{\alpha_{t}}\left(\bar{x}_{t}-x_{*}\right)-\frac{A_{t-1}}{\alpha_{t}}\left(\bar{x}_{t-1}-x_{*}\right), \nabla f\left(\bar{x}_{t}\right)\right\rangle \qquad \left(-x_*= \frac{A_{t-1}}{\alpha_t}x_* - \frac{A_{t}}{\alpha_t}x_*\right)\\
&=\sum_{t=1}^{T} A_{t}\left\langle\bar{x}_{t}-x_{*}, \nabla f\left(\bar{x}_{t}\right)\right\rangle-A_{t-1}\left\langle\bar{x}_{t-1}-x_{*}, \nabla f\left(\bar{x}_{t}\right)\right\rangle \\
&=\sum_{t=1}^{T}(\alpha_t+A_{t-1})\left\langle\bar{x}_{t}-x_{*}, \nabla f\left(\bar{x}_{t}\right\rangle\right)-A_{t-1}\left\langle\bar{x}_{t-1}-x_{*}, \nabla f\left(\bar{x}_{t}\right)\right\rangle\\
&=\sum_{t=1}^{T}\alpha_{t}\left\langle\bar{x}_{t}-x_{*}, \nabla f\left(\bar{x}_{t}\right)\right\rangle+A_{t-1}\left\langle\bar{x}_{t}-\bar{x}_{t-1}, \nabla f\left(\bar{x}_{t}\right)\right\rangle\\
& \geq \sum_{t=1}^{T} \alpha_{t}\left(f\left(\bar{x}_{t}\right)-f\left(x_{*}\right)\right)+\sum_{t=1}^{T} \sum_{i=1}^{t-1} \alpha_{i}\left(f\left(\bar{x}_{t}\right)-f\left(\bar{x}_{t-1}\right)\right) \qquad \left( f(x)-f(y) \leq \nabla f(x)^\top(x-y) \right)
\end{aligned}
\end{equation*}
where in the last line we use $f(y) \geq f(x) + \nabla f(x)^\top(y-x)$ and multiply by $(-1)$. We also take $\alpha_0=0$ and $A_0 = 0$. Then, we telescope the double summation and reorganize the terms

\begin{equation*}
\begin{aligned}
&=\sum_{t=1}^{T} \alpha_{t}\left(f\left(\bar{x}_{t}\right)-f\left(x_{*}\right)\right)+\sum_{t=1}^{T-1} \alpha_{t}\left(f\left(\bar{x}_{T}\right)-f\left(\bar{x}_{t}\right)\right) \\
&=\alpha_{T}\left(f\left(\bar{x}_{T}\right)-f\left(x_{*}\right)\right)+\sum_{t=1}^{T-1} \alpha_{t}\left(f\left(\bar{x}_{t}\right)-f\left(x_{*}\right)+f\left(\bar{x}_{T}\right)-f\left(\bar{x}_{t}\right)\right) \\
&=\sum_{t=1}^{T} \alpha_{t}\left(f\left(\bar{x}_{T}\right)-f\left(x_{*}\right)\right)
\end{aligned}
\end{equation*}

Having simplified the expression, we divide both sides by $A_{T}$ and conclude the proof. Observe that $A_{T} \geq \frac{T^{2}}{2},$ hence,

\begin{equation*}
\begin{aligned}
\sum_{t=1}^{T} \alpha_{t}\left(f\left(\bar{x}_{T}\right)-f\left(x_{*}\right)\right) & \leq \sum_{t=1}^{T} \alpha_{t}\left\langle x_{t}-x_{*}, \nabla f\left(\bar{x}_{t}\right)\right\rangle \\
\frac{1}{A_{T}} \sum_{t=1}^{T} \alpha_{t}\left(f\left(\bar{x}_{T}\right)-f\left(x_{*}\right)\right) & \leq \frac{1}{A_{T}} \sum_{t=1}^{T} \alpha_{t}\left\langle x_{t}-x_{*}, \nabla f\left(\bar{x}_{t}\right)\right\rangle \\
f\left(\bar{x}_{T}\right)-f\left(x_{*}\right) & \leq \frac{2 R_{T}\left(x_{*}\right)}{T^{2}}
\end{aligned}
\end{equation*}	
\end{proof}	
\end{lemma}

\subsection{Non-smooth Setting}
Before starting with the mathematical analysis of the convergence rate of UnixGrad in the non-smooth setting, we recall some useful lemmas.

\begin{lemma}
\label{ineq:alphas}
	Let $\left\{a_{i}\right\}_{i=1, \ldots, n}$ be a sequence of non negative numbers. Then, it holds that
$$
\sqrt{\sum_{i=1}^{n} a_{i}} \leq \sum_{i=1}^{n} \frac{a_{i}}{\sum_{j=1}^{i} a_{j}} \leq 2 \sqrt{\sum_{i=1}^{n} a_{i}}
$$
\begin{proof}
	Please refer to \cite{accelegrad} for the proof.
\end{proof}
\end{lemma}

\begin{lemma}

\begin{equation}
\label{eq:young}
	\begin{aligned}
		\alpha_{t}\left\|g_{t}-M_{t}\right\|_{*}\left\|x_{t}-y_{t}\right\|=\inf _{\rho>0}\left\{\frac{\rho}{2}\left\|g_{t}-M_{t}\right\|_{*}^{2}+\frac{\alpha_{t}^{2}}{2 \rho}\left\|x_{t}-y_{t}\right\|^{2}\right\}
	\end{aligned}
\end{equation}

\begin{proof}
	Given two nonnegative real numbers $a\geq 0, b\geq 0 $ and two real number $p>0$ and $q>0$ such that $1/p + 1/q = 1$, then 
	$$
		ab \leq \dfrac{a^p}{p} + \dfrac{b^q}{q}. \qquad \text{(Young's inequality)}
	$$
	By using it with $p,q=2$ we have that
	$$
		\alpha_{t}\left\|g_{t}-M_{t}\right\|_{*}\left\|x_{t}-y_{t}\right\| \leq \frac{1}{2}\left\|g_{t}-M_{t}\right\|_{*}^{2}+\frac{\alpha_{t}^{2}}{2 }\left\|x_{t}-y_{t}\right\|^{2}.
	$$
	Now, since the equality holds when $a^p=b^q$ we can use mathematical trick and formulate the statement as
	$$
		\alpha_{t}\left\|g_{t}-M_{t}\right\|_{*}\left\|x_{t}-y_{t}\right\|=\inf _{\rho>0}\left\{\frac{\rho}{2}\left\|g_{t}-M_{t}\right\|_{*}^{2}+\frac{\alpha_{t}^{2}}{2 \rho}\left\|x_{t}-y_{t}\right\|^{2}\right\}.
	$$
\end{proof}
\end{lemma}

\subsubsection{Deterministic setting}
\begin{theorem}
	Consider the constrained optimization setting, where $f: \mathcal{K} \rightarrow \mathbb{R}$ is a proper, convex and $G$-Lipschitz function defined over compact, convex set $\mathcal{K}$. Let $x^{*} \in \min _{x \in \mathcal{K}} f(x)$. Then, UnixGrad algorithm guarantees
$$
f\left(\bar{x}_{T}\right)-\min _{x \in \mathcal{K}} f(x) \leq \frac{7 D \sqrt{1+\sum_{t=1}^{T} \alpha_{t}^{2}\left\|g_{t}-M_{t}\right\|_{*}^{2}}-D}{T^{2}} \leq \frac{6 D}{T^{2}}+\frac{14 G D}{\sqrt{T}}
$$

\begin{proof}
$$
\sum_{t=1}^{T} \alpha_{t}\left\langle x_{t}-x_{*}, g_{t}\right\rangle=\sum_{t=1}^{T} \underbrace{\alpha_{t}\left\langle x_{t}-y_{t}, g_{t}-M_{t}\right\rangle}_{(\mathrm{A})}+\underbrace{\alpha_{t}\left\langle x_{t}-y_{t}, M_{t}\right\rangle}_{(\mathrm{B})}+\underbrace{\alpha_{t}\left\langle y_{t}-x^{*}, g_{t}\right\rangle}_{(\mathrm{C})}
$$

\textbf{Bounding (A)}
$$
\begin{aligned}
\sum_{t=1}^{T} \alpha_{t}\left\langle x_{t}-y_{t}, g_{t}-M_{t}\right\rangle & \leq \sum_{t=1}^{T} \alpha_{t}\left\|g_{t}-M_{t}\right\|_{*}\left\|x_{t}-y_{t}\right\| \quad \text{ (H�lder's Inequality) } \\
& \leq \sum_{t=1}^{T} \frac{\rho}{2}\left\|g_{t}-M_{t}\right\|_{*}^{2}+\frac{\alpha_{t}^{2}}{2 \rho}\left\|x_{t}-y_{t}\right\|^{2} \quad \text { (Equation (\ref{eq:young})) }
\end{aligned}
$$
By setting $\rho=\alpha_{t}^{2} \eta_{t+1}\,\,(\alpha_t, \eta_{t+1}>0)$, we get the following upper bound for term (A),
$$
\sum_{t=1}^{T} \alpha_{t}\left\langle x_{t}-y_{t}, g_{t}-M_{t}\right\rangle \leq \sum_{t=1}^{T} \frac{\alpha_{t}^{2} \eta_{t+1}}{2}\left\|g_{t}-M_{t}\right\|_{*}^{2}+\frac{1}{2 \eta_{t+1}}\left\|x_{t}-y_{t}\right\|^{2}
$$

\textbf{Bounding (B)}\\
In order to find an upper-bound for this term, we have to exploit the first order optimality condition of $x_t$, that is $\nabla f(x_t)^\top(y-x_t) \geq 0, \, \forall y\, \text{feasible}$ (all feasible directions from $x$ are aligned with increasing gradient). Since $\nabla f(x_t) = \alpha_t M_t + \frac{1}{\eta_t}\nabla_x \mathcal{D}_{\mathcal{R}}(x_t, y_{t-1})$, then 
$$ 
	\alpha_t \left\langle y_{t}-x_{t}, M_{t}\right\rangle + \frac{1}{\eta_t}\nabla_x \mathcal{D}_{\mathcal{R}}(x_t, y_{t-1})^\top (y_{t}-x_{t}) \geq 0.
$$
By simply rearranging the above inequality we get
$$
	\frac{1}{\eta_t}\nabla_x \mathcal{D}_{\mathcal{R}}(x_t, y_{t-1})^\top (y_{t}-x_{t}) \geq \alpha_t \left\langle x_{t} - y_{t}, M_{t}\right\rangle.
$$


Now the upper-bound becomes straightforward.
$$
\begin{aligned}
\sum_{t=1}^{T} \alpha_{t}\left\langle x_{t}-y_{t}, M_{t}\right\rangle & \leq \sum_{t=1}^{T} \frac{1}{\eta_{t}} \nabla_{x} D_{\mathcal{R}}\left(x_{t}, y_{t-1}\right)^{T}\left(y_{t}-x_{t}\right) \quad \text { (Optimality for } \left.x_{t}\right) \\
&=\sum_{t=1}^{T} \frac{1}{\eta_{t}}\left(D_{\mathcal{R}}\left(y_{t}, y_{t-1}\right)-D_{\mathcal{R}}\left(x_{t}, y_{t-1}\right)-D_{\mathcal{R}}\left(y_{t}, x_{t}\right)\right)
\end{aligned}
$$

\textbf{Bounding (C)}\\
Following the same procedure used in the previous bound we find the upper-bound also for the third term.
$$
\begin{aligned}
\sum_{t=1}^{T} \alpha_{t}\left\langle y_{t}-x^{*}, g_{t}\right\rangle & \leq \sum_{t=1}^{T} \frac{1}{\eta_{t}} \nabla_{x} D_{\mathcal{R}}\left(y_{t}, y_{t-1}\right)^{T}\left(x^{*}-y_{t}\right) \quad \text { (Optimality for } \left.y_{t}\right) \\
&=\sum_{t=1}^{T} \frac{1}{\eta_{t}}\left(D_{\mathcal{R}}\left(x^{*}, y_{t-1}\right)-D_{\mathcal{R}}\left(y_{t}, y_{t-1}\right)-D_{\mathcal{R}}\left(x^{*}, y_{t}\right)\right)
\end{aligned}
$$

\textbf{Final Bound}
$$
\begin{aligned}
\sum_{t=1}^{T} \alpha_{t}\left\langle x_{t}-x_{*}, g_{t}\right\rangle \leq & \sum_{t=1}^{T} \frac{\alpha_{t}^{2} \eta_{t+1}}{2}\left\|g_{t}-M_{t}\right\|_{*}^{2}+\frac{1}{2 \eta_{t+1}}\left\|x_{t}-y_{t}\right\|^{2} \\
&+\frac{1}{\eta_{t}}\left(D_{\mathcal{R}}\left(x^{*}, y_{t-1}\right)-D_{\mathcal{R}}\left(x^{*}, y_{t}\right)-D_{\mathcal{R}}\left(x_{t}, y_{t-1}\right)-D_{\mathcal{R}}\left(y_{t}, x_{t}\right)\right) \\
\leq & \sum_{t=1}^{T} \frac{\alpha_{t}^{2} \eta_{t+1}}{2}\left\|g_{t}-M_{t}\right\|_{*}^{2}+\frac{1}{2 \eta_{t+1}}\left\|x_{t}-y_{t}\right\|^{2} \\
&+\frac{1}{\eta_{t}}\left(D_{\mathcal{R}}\left(x^{*}, y_{t-1}\right)-D_{\mathcal{R}}\left(x^{*}, y_{t}\right)-\frac{1}{2}\left(\left\|x_{t}-y_{t}\right\|^{2}+\left\|x_{t}-y_{t-1}\right\|^{2}\right)\right) \\
\leq & \sum_{t=1}^{T} \frac{\alpha_{t}^{2} \eta_{t+1}}{2}\left\|g_{t}-M_{t}\right\|_{*}^{2}+\sum_{t=1}^{T-1}\left(\frac{1}{\eta_{t+1}}-\frac{1}{\eta_{t}}\right) D_{\mathcal{R}}\left(x^{*}, y_{t}\right) \\
&+\sum_{t=1}^{T}\left(\frac{1}{\eta_{t+1}}-\frac{1}{\eta_{t}}\right)\left\|x_{t}-y_{t}\right\|^{2}+\frac{1}{\eta_{1}} D^{2} \\
\leq & \sum_{t=1}^{T} \frac{\alpha_{t}^{2} \eta_{t+1}}{2}\left\|g_{t}-M_{t}\right\|_{*}^{2}+\sum_{t=1}^{T}\left(\frac{1}{\eta_{t+1}}-\frac{1}{\eta_{t}}\right)\left\|x_{t}-y_{t}\right\|^{2}+\frac{D^{2}}{\eta_{T}}+\frac{D}{2} \\
\leq & \sum_{t=1}^{T} \frac{\alpha_{t}^{2} \eta_{t+1}}{2}\left\|g_{t}-M_{t}\right\|_{*}^{2}+D^{2}\left(\frac{2}{\eta_{T+1}}+\frac{1}{\eta_{T}}\right)+\frac{D}{2} \\
& \leq D \sum_{t=1}^{T} \frac{\alpha_{t}^{2}\left\|g_{t}-M_{t}\right\|_{*}^{2}}{\sqrt{1+\sum_{i=1}^{t} \alpha_{t}^{2}\left\|g_{t}-M_{t}\right\|_{*}^{2}}}+\frac{3}{2} D \sqrt{1+\sum_{t=1}^{T} \alpha_{t}^{2}\left\|g_{t}-M_{t}\right\|_{*}^{2}+\frac{D}{2}} \\
& \leq \frac{7}{2} D \sqrt{1+\sum_{t=1}^{T} \alpha_{t}^{2}\left\|g_{t}-M_{t}\right\|_{*}^{2}-\frac{D}{2}} \\
& \leq 3 D+7 G D \sqrt{\sum_{t=1}^{T} \alpha_{t}^{2}} \\
& \leq 3 D+7 G D T^{3 / 2}
\end{aligned}
$$
We obtain the rate by applying Lemma \ref{regret} to the weighted regret bound above
\end{proof}

\end{theorem}




\end{appendices}


\end{document}
